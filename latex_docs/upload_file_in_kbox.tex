\documentclass{article}
\usepackage[utf8]{inputenc}
\usepackage{amsmath,amssymb}
\usepackage{geometry}
\geometry{a4paper, margin=1in}

\begin{document}

\section*{Costo di Caricamento e Indicizzazione di File e Media}

Quando un utente vuole aggiungere documenti (PDF, DOC, immagini, video, ecc.) nella propria \emph{Knowledge Box} (KBox) o \emph{Vector Store} (MongoDB Atlas), l'applicazione esegue diversi \textbf{passi}:

\begin{enumerate}
    \item \textbf{Processamento iniziale} del file (lettura, estrazione testo, eventuale OCR o captioning se è un’immagine/video).
    \item \textbf{Chunking} (divisione in blocchi di lunghezza moderata per migliorare la ricerca).
    \item \textbf{Calcolo degli embedding} dei chunk (con modello dedicato di embedding).
    \item \textbf{Salvataggio} dei chunk (testo e metadata) nel DB e \textbf{salvataggio} dei relativi vettori embedding nel Vector Store.
\end{enumerate}

Ciascuno di questi passi introduce un \textbf{costo}. Di seguito lo \emph{parametrizziamo} e poi presentiamo un \emph{esempio} per diversi tipi di file: PDF in pipeline Hi-Res, documento testuale standard (pipeline Fast), immagine e video.

\section*{1. Parametri e Formule Generali}

Definiamo una \textbf{formula generale} per il caricamento di un file (o media) che abbia:

\begin{itemize}
    \item \(N_{\text{pages}}\): numero di pagine (se testo/PDF). Oppure, se immagine, consideriamo 1 ``pagina''; se video, potrà essere equiparato a un certo numero di frame/pagine elaborate.
    \item \(C_{\text{processing}}\): costo di \emph{processing} ``Unstructured'' (o analogo) per estrarre il testo (oppure, nel caso di immagini/video, la parte di OCR/captioning).
    \item \(C_{\text{embedding}}\): costo di generazione embedding per i chunk ottenuti.
    \item \(C_{\text{DB}}\): costo di scrittura dei chunk e dei vettori embedding su MongoDB Atlas.
\end{itemize}

\subsection*{1.1 Costo di Processamento (Unstructured)}
Assumiamo due pipeline tipiche:

\begin{enumerate}
    \item \textbf{Fast Pipeline}: \$0,001 per pagina
    \item \textbf{Hi-Res Pipeline}: \$0,01 per pagina
\end{enumerate}

\emph{(Fonte: Unstructured.io, oppure AWS Marketplace equivalenti)}

\paragraph{Formula di base}
\[
C_{\text{processing}}
=
N_{\text{pages}} \times c_{\text{page}},
\]
dove \(c_{\text{page}} \in \{0,001, \; 0,01\}\) a seconda di Fast o Hi-Res.

\subsection*{1.2 Chunking}
Il file di testo (ottenuto da PDF, DOC, o da un OCR) viene suddiviso in blocchi di lunghezza \(\bar{T}_{\text{chunk}}\) token. Se il testo complessivo di un file è \(T_{\text{total}}\) token, il \textbf{numero di chunk}:
\[
N_{\text{chunk}}
=
\left\lceil
\frac{T_{\text{total}}}{\bar{T}_{\text{chunk}}}
\right\rceil.
\]
\emph{(In alcuni casi, 1 chunk per pagina se $\sim$500 token/pagina.)}

\subsection*{1.3 Calcolo degli Embedding}
Per ogni chunk si genera un embedding (es.\ con \emph{text-embedding-3-small}).
\begin{itemize}
    \item \textbf{Costo} per 1k token = \$0,00002 (o meno).
\end{itemize}
Se un chunk tipico ha \(\bar{T}_{\text{chunk}}\) token, allora il \textbf{costo embedding} di 1 chunk:
\[
C_{\text{embed\_chunk}}
=
\frac{\bar{T}_{\text{chunk}}}{1000}
\;\times\;
c_{\text{embed}}.
\]
Moltiplicando per $N_{\text{chunk}}$:
\[
C_{\text{embedding}}
=
N_{\text{chunk}}
\;\times\;
C_{\text{embed\_chunk}}.
\]

\subsection*{1.4 Costo di Scrittura DB e Vector Store}
\begin{itemize}
    \item Ogni chunk (testo + metadata) si inserisce in una collezione DB, con un costo di scrittura $\approx d3$.
    \item Ogni embedding (es.\ 1536 dimensioni) occupa $\sim 6$KB: scrittura su Vector Store con $\approx 6$ WPU.
\end{itemize}
\[
C_{\text{DB\_1\_chunk}}
=
\underbrace{\frac{1}{10^6}\times C_{\text{WPU}}}_{\text{testo}}
\;+\;
\underbrace{\frac{\text{embedding\_KB}}{10^6}\times C_{\text{WPU}}}_{\text{vettore}}
\]
Se \(\text{embedding\_KB} \approx 6\) e \(C_{\text{WPU}} \approx 1,25\) \$, allora scrivere 1 chunk con embedding costerà $\approx 7{,}5\times 10^{-6}\$.

Per semplicità:
\[
C_{\text{DB}}
=
N_{\text{chunk}}
\;\times\;
C_{\text{DB\_1\_chunk}}.
\]

\subsection*{1.5 Formula di Caricamento di un File}
Combinando il tutto:

\[
\boxed{
C_{\text{upload\_file}}
=
C_{\text{processing}}
\;+\;
C_{\text{embedding}}
\;+\;
C_{\text{DB}}
}
\]
dove:
\[
\begin{aligned}
C_{\text{processing}} &= N_{\text{pages}} \times c_{\text{page}}, \\[6pt]
N_{\text{chunk}} &= \left\lceil \frac{T_{\text{total}}}{\bar{T}_{\text{chunk}}} \right\rceil, \\[6pt]
C_{\text{embedding}}
&= N_{\text{chunk}}
\;\times\;
\frac{\bar{T}_{\text{chunk}}}{1000}\, c_{\text{embed}},\\[6pt]
C_{\text{DB}}
&= N_{\text{chunk}}
\;\times\;
C_{\text{DB\_1\_chunk}}.
\end{aligned}
\]

\section*{2. Caso di Documenti PDF (Hi-Res vs.\ Standard)}

\subsection*{2.1 Caso ``PDF con Pipeline Hi-Res''}
\begin{itemize}
    \item \(c_{\text{page}} = 0,01\$ (\emph{Hi-Res}).
    \item Esempio: $N_{\text{pages}}=10$.
\end{itemize}
\[
C_{\text{processing}}
=
10 \;\times\; 0{,}01 = 0{,}10\$\quad
(\text{= 10 cent totali}).
\]
Se ogni pagina $\sim 500$ token, allora $T_{\text{total}}=5000$ token su 10 pagine, e con chunk da 500 token: $N_{\text{chunk}}=10$.

\paragraph{Embedding}
\[
C_{\text{embed\_chunk}}
=
\frac{500}{1000} \;\times\; 0{,}00002
=
0{,}00001\$ \quad(1\times 10^{-5}).
\]
Dieci chunk $\to 10 \;\times 10^{-5} = 10^{-4}\$.

\paragraph{DB}
Scrivere 1 chunk con embedding $\approx 7,5\times10^{-6}\$; per 10 chunk: $7,5\times10^{-5}\$.

\[
C_{\text{upload\_file}}
=
0{,}10 + 0{,}0001 + 0{,}000075
\approx
0{,}100175
\approx
0{,}10\$.
\]
\emph{Quasi tutto dovuto alla pipeline Hi-Res.}

\subsection*{2.2 Caso ``Documento Standard'' (Fast Pipeline)}
\begin{itemize}
    \item $c_{\text{page}}=0{,}001\$ (\emph{Fast}).
    \item Esempio: 10 pagine $\to 10\times0,001 = 0,01\$ di processing$.
\end{itemize}

Stessa logica di embedding e DB: $0,0001 + 0,000075 = 0,000175\$.

\[
C_{\text{upload\_file}}
\approx
0{,}01 + 0{,}000175 = 0{,}010175
\approx
0{,}0102\$.
\]

\section*{3. Caso di File Multimediali (Immagini e Video)}

\noindent
\textbf{Immagini:} potremmo usare GPT-4o (o GPT-4o Mini) per ricavare una descrizione testuale (``caption''). L'immagine diventa input visivo, convertito in token.\\
\textbf{Video:} estraiamo alcuni frame, ognuno trattato come un'immagine.

\subsection*{3.1 Caricamento di un’Immagine}

\paragraph{Passi:}
\begin{enumerate}
    \item \textbf{Caption}: chiamata LLM in input con $\approx T_{\text{img}}$ token (dipende da dimensione, es.\ 512$\times$512 $\to$ 255 token).
    \item \textbf{Output}: testo di descrizione (es.\ 50--100 token).
    \item \textbf{Embedding}: dei $\sim 100$ token descrizione.
    \item \textbf{Scrittura}: salviamo descrizione e embedding nel DB.
\end{enumerate}

\paragraph{Formula}
\[
C_{\text{upload\_img}}
=
C_{\text{LLM\_caption}}
+ C_{\text{embed}}
+ C_{\text{DB}},
\]
dove
\[
C_{\text{LLM\_caption}}
=
\frac{T_{\text{img}}}{1000}\,p_{\text{in}}
+
\frac{T_{\text{descr}}}{1000}\,p_{\text{out}},
\]
\[
C_{\text{embed}}
\approx
\frac{T_{\text{descr}}}{1000}\times c_{\text{embed}},\quad
C_{\text{DB}}
\approx 7,5\times10^{-6}\$ \text{ (per embedding da 6KB)}.
\]

\paragraph{Esempio} (con GPT-4o)
\[
T_{\text{img}}=255,\quad
T_{\text{descr}}=100,\quad
p_{\text{in}}=0{,}005,\quad
p_{\text{out}}=0{,}015.
\]
\[
C_{\text{LLM\_caption}}
=
\frac{255}{1000}\times0{,}005
+
\frac{100}{1000}\times0{,}015
=
0{,}001275+0{,}0015
=
0{,}002775\$.
\]
Embedding 100 token $\to 2\times10^{-6}\$; DB scrittura $\sim7,5\times10^{-6}\$.
Totale $\approx 0{,}00278\$ ($\sim 0,28 cent).

\textbf{(Se GPT-4o Mini)}, costi scendono di ~30$\times$, con input 255$\times$0,00015 + output 100$\times$0,00060 $\approx 0,00009825\$.

\subsection*{3.2 Caricamento di un Video}

\paragraph{Passi:}
\begin{enumerate}
    \item \textbf{Estrazione Frame}: supponiamo di estrarre 1 frame ogni X secondi, generando $N_{\text{frame}}$ immagini.
    \item \textbf{Caption} per ciascun frame, come se fosse un'immagine.
    \item \textbf{Embedding}: potremmo unire le caption in un testo unico, generando un embedding, o creare embedding per ogni frame.
    \item \textbf{Scrittura}: su DB.
\end{enumerate}

\paragraph{Formula}
\[
C_{\text{video}}
\approx
N_{\text{frame}}
\times
C_{\text{LLM\_frame}}
+
C_{\text{embed}}
+
C_{\text{DB}}.
\]

\paragraph{Esempio}
\begin{itemize}
    \item Video 2 min, estraiamo 1 frame/10 s $\to N_{\text{frame}}=12$.
    \item GPT-4o per caption di ogni frame (ipotizziamo $\sim0{,}0027\$) $\to 12\times0{,}0027=0{,}0324\$.
    \item Testo totale 12 caption $\times$ 50 token = 600 token $\to$ embedding $=600/1000\times0{,}00002=1{,}2\times10^{-5}\$.
    \item DB scrittura embedding $\sim7{,}5\times10^{-6}\$.
\end{itemize}
Totale $\approx 0{,}032417\$ (\sim 3,24 cent). \textbf{(Se GPT-4o Mini)}, costi scendono di un fattore ~10--30.

\section*{4. Conclusioni ed Esempi Riassuntivi}

\textbf{Caricamento di documenti} (testuali):
\begin{itemize}
    \item \emph{Hi-Res} (\$0,01/pagina) domina i costi (\$0,10 per 10 pagine).
    \item \emph{Fast} (\$0,001/pagina) scende a millesimi.
\end{itemize}

\textbf{Caricamento di immagini}:
\begin{itemize}
    \item Principale costo se usiamo GPT-4o per caption (fino a ~0,002--0,003\$ per immagine).
    \item Embedding e DB microcentesimi.
    \item GPT-4o Mini: $\sim0,0001\$ per immagine.
\end{itemize}

\textbf{Caricamento di video}:
\begin{itemize}
    \item Diviso in ``frame extraction + captioning'' (nessun cenno ad altro).
    \item Esempio 2 min, 12 frame, GPT-4o $\approx0,0324\$ tot (caption). Embedding e DB trascurabili. GPT-4o Mini riduce di ~20$\times$.
\end{itemize}

\paragraph{Formula Generale} per un file (testo o media) caricabile:
\[
\boxed{
C_{\text{upload\_file}}
=
C_{\text{process}}
+
N_{\text{chunk}}
\times
\biggl[
 \frac{\bar{T}_{\text{chunk}}}{1000} \, c_{\text{embed}}
 + C_{\text{DB\_chunk}}
\biggr]
+
(\text{costo extra se immagine/video}).
}
\]
Dove
\begin{itemize}
    \item $C_{\text{process}} = N_{\text{pages}} \times c_{\text{page}}$ (0,001\$ o 0,01\$).
    \item $N_{\text{frame}} = \frac{\text{durata\_s}}{\text{sampling\_sec}}$ per i video.
    \item $C_{\text{LLM\_caption}} = \Bigl(\frac{T_{\text{img}}}{1000}\Bigr) p_{\text{in}} + \Bigl(\frac{T_{\text{descr}}}{1000}\Bigr) p_{\text{out}}$ per ogni immagine/frame.
\end{itemize}

\subsection*{Esempio Finale Riassuntivo}

\begin{center}
\begin{tabular}{|l|c|c|c|c|c|c|}
\hline
\textbf{Tipo File} & \textbf{Pagine/Frame} & \textbf{Pipeline/Modello} & \textbf{Costo Process.} & \textbf{Costo LLM} & \textbf{Embedding + DB} & \textbf{Totale}\\
\hline
\textbf{PDF 10 pag (Hi-Res)} & 10 & 0,01\$/pagina & 0,10\$ & -- & $\sim0,000175\$ & 0,10018\$ \\ \hline
\textbf{PDF 10 pag (Fast)}   & 10 & 0,001\$/pagina & 0,01\$ & -- & $\sim0,000175\$ & 0,01018\$ \\ \hline
\textbf{Immagine (GPT-4o)}   & 1 & $(p_{\text{in}}=0,005; p_{\text{out}}=0,015)$ & -- & $\sim0,0028\$ & $<0,00002\$ & 0,00282\$ \\ \hline
\textbf{Immagine (Mini)}     & 1 & $(p_{\text{in}}=0,00015;p_{\text{out}}=0,00060)$ & -- & $\sim0,0001\$ & $<0,00002\$ & 0,00012\$ \\ \hline
\textbf{Video 2 min (GPT-4o)} & 12 frame & $\sim0,0027\$ \text{ x frame}$ & -- & $12\times0,0027=0,0324\$ & $\sim0,00002\$ & $\sim0,03242\$ \\ \hline
\textbf{Video 2 min (Mini)}   & 12 frame & $\sim0,0001\$ \text{ x frame}$ & -- & $12\times0,0001=0,0012\$ & $\sim0,00002\$ & $\sim0,00122\$ \\ \hline
\end{tabular}
\end{center}

\medskip
A
Come si vede:
\begin{itemize}
    \item \emph{PDF} con pipeline Hi-Res: \$0,10\ldots\$0,20 in base al numero di pagine (qui 10).
    \item \emph{PDF} con pipeline Fast: 10$\times$ meno.
    \item \emph{Immagini e video}: costo imputabile quasi tutto alle \emph{chiamate LLM} (GPT-4o o GPT-4o Mini) per ottenere le descrizioni. Embedding + DB restano microcentesimi.
\end{itemize}

\textbf{In conclusione}, il \emph{costo di caricamento} di file (testo/immagini/video) è spesso dominato:
\begin{itemize}
    \item Dai \textbf{cost per pagina} (se si usa SaaS Unstructured).
    \item Dalle \textbf{chiamate GPT-4o} (se facciamo caption su immagini/video).
    \item L'\textbf{embedding} e DB incidono in microcentesimi.
\end{itemize}
Pertanto, un PDF grande con pipeline Hi-Res rimane su alcuni centesimi (0,01\$ $\times$ pagine), mentre un PDF standard (Fast) scende a millesimi. Per le immagini e i video \emph{dipende dal modello LLM} usato per la descrizione: GPT-4o può costare $\sim0,003\$$/immagine, GPT-4o Mini $\sim0,0001\$$. \\
Lo \textbf{storage} finale su MongoDB inciderà poi su base mensile (0,25\$/GB), se l'utente carica molti MB/GB di contenuti multimediali.

\end{document}
